\documentclass[a4paper,12pt]{article}
\usepackage{amsmath}
\usepackage{graphicx}
\usepackage{booktabs}
\usepackage{geometry}
\usepackage{hyperref}

\geometry{margin=1in}

\title{ Design and Analysis of 2-Link Robotic Manipulator}
\author{Task 4}
\date{}

\begin{document}

\maketitle

\section*{Introduction}
This document presents the structured design and analysis of a 2-link robotic manipulator capable of lifting small objects weighing between 50g and 200g. It is organized into key sections that sequentially develop the theory and practical considerations for kinematics, torque, actuation, overload behavior, and efficiency.

\section*{Index}
\begin{enumerate}
    \item \hyperref[sec:kinematics]{Kinematics (Forward and Inverse)}
    \item \hyperref[sec:torque]{Torque Estimation for Joint Motion}
    \item \hyperref[sec:motors]{Motor Selection Based on Torque}
    \item \hyperref[sec:overload]{Effect of Unexpected Overload}
    \item \hyperref[sec:efficiency]{Energy Efficiency Enhancements}
\end{enumerate}
\vspace{1em}
\noindent\rule{\linewidth}{0.6pt}
\vspace{1em}
\section{Kinematics (Forward and Inverse)} \label{sec:kinematics}

The 2-link robotic manipulator under consideration consists of two revolute joints, denoted by joint angles $q_1$ and $q_2$, and corresponding rigid links of lengths $l_1$ and $l_2$. This section presents both forward and inverse kinematic formulations, enabling the computation of the end-effector's position in the workspace and the joint angles required to reach a specific point, respectively.

\subsection*{Reference Frame and Joint Definitions}

We define a global inertial frame $\{O\}$ with its origin located at the base of the manipulator. The $x$-axis points horizontally to the right, and the $y$-axis points vertically upwards.

\begin{itemize}
    \item $q_1$: The angle between the first link and the global $x$-axis, measured counterclockwise.
    \item $q_2$: The relative angle between the second link and the first link, also measured counterclockwise.
\end{itemize}

The absolute orientation of the second link is thus given by $\theta_2 = q_1 + q_2$.

\subsection*{Forward Kinematics}

To determine the Cartesian coordinates $(x, y)$ of the end-effector, we use the following parametric equations derived from trigonometric projections of the links:

\begin{align}
    x &= l_1 \cos(q_1) + l_2 \cos(q_1 + q_2) \\
    y &= l_1 \sin(q_1) + l_2 \sin(q_1 + q_2)
\end{align}

These equations assume ideal planar motion with no joint or link flexibility.

\subsection*{Inverse Kinematics}

Given a desired position $(x, y)$ within the reachable workspace, the inverse kinematics problem involves computing the corresponding joint angles $(q_1, q_2)$.

We begin by solving for $q_2$ using the Law of Cosines:

\begin{align}
    \cos(q_2) &= \frac{x^2 + y^2 - l_1^2 - l_2^2}{2 l_1 l_2} \\
    q_2 &= \arccos\left(\cos(q_2)\right)
\end{align}

Then, using geometric construction and trigonometric identities:

\begin{align}
    q_1 &= \arctan2(y, x) - \arctan2\left(l_2 \sin(q_2), l_1 + l_2 \cos(q_2)\right)
\end{align}

Note that two possible configurations (elbow-up and elbow-down) may exist for a single end-effector position. The $\texttt{atan2}$ function ensures proper quadrant handling for accurate angle resolution.

\vspace{1em}
\noindent\rule{\linewidth}{0.6pt}
\vspace{1em}
\section{Torque Estimation for Joint Motion} \label{sec:torque}

To ensure accurate actuation and safe operation of the robotic arm, we derive the full dynamic equations of motion using the Euler-Lagrange formulation. The resulting expressions provide the torques $\tau_1$ and $\tau_2$ required at joints 1 and 2, respectively, accounting for link inertia, Coriolis effects, and gravity. Additionally, the payload mass $m_p$ held at the end-effector is incorporated into the model.

\subsection*{System Parameters}
\begin{itemize}
    \item Link 1: mass $m_1$, length $l_1$, moment of inertia $I_1$ (about joint 1)
    \item Link 2: mass $m_2$, length $l_2$, moment of inertia $I_2$ (about joint 2)
    \item Payload at end-effector: mass $m_p$ (in range 0.05–0.2 kg)
\end{itemize}

We assume the links are uniform rods:
\[
I_1 = \frac{1}{3} m_1 l_1^2, \quad I_2 = \frac{1}{3} m_2 l_2^2
\]

\subsection*{Kinetic Energy}

The total kinetic energy $T$ of the system includes:
\begin{itemize}
    \item Rotational energy of both links
    \item Translational kinetic energy of their centers of mass
    \item Translational kinetic energy of the payload
\end{itemize}

\[
T = \frac{1}{2} I_1 \dot{q}_1^2 
+ \frac{1}{2} m_1 v_{c1}^2 
+ \frac{1}{2} I_2 (\dot{q}_1 + \dot{q}_2)^2 
+ \frac{1}{2} m_2 v_{c2}^2 
+ \frac{1}{2} m_p v_e^2
\]

\noindent where:
\begin{itemize}
    \item $v_{c1}, v_{c2}$ are the velocities of the link centers of mass
    \item $v_e$ is the velocity of the end-effector
\end{itemize}

\subsection*{Potential Energy}

Let $g$ be gravitational acceleration (typically $9.81~\text{m/s}^2$). The total potential energy is:

\[
V = m_1 g y_{c1} + m_2 g y_{c2} + m_p g y_e
\]

\noindent where:
\begin{itemize}
    \item $y_{c1} = \frac{l_1}{2} \sin(q_1)$
    \item $y_{c2} = l_1 \sin(q_1) + \frac{l_2}{2} \sin(q_1 + q_2)$
    \item $y_e = l_1 \sin(q_1) + l_2 \sin(q_1 + q_2)$
\end{itemize}

\subsection*{Dynamic Model via Lagrangian}

The Lagrangian is:
\[
\mathcal{L} = T - V
\]

We then compute the joint torques using the Euler-Lagrange equations:
\[
\frac{d}{dt} \left( \frac{\partial \mathcal{L}}{\partial \dot{q}_i} \right) - \frac{\partial \mathcal{L}}{\partial q_i} = \tau_i \quad \text{for } i = 1, 2
\]

\subsection*{Resulting Torque Equations}

The full dynamic model can be written in matrix form as:

\[
\begin{bmatrix}
\tau_1 \\
\tau_2
\end{bmatrix}
=
\mathbf{M}(q)
\begin{bmatrix}
\ddot{q}_1 \\
\ddot{q}_2
\end{bmatrix}
+
\mathbf{C}(q, \dot{q})
\begin{bmatrix}
\dot{q}_1 \\
\dot{q}_2
\end{bmatrix}
+
\mathbf{G}(q)
\]

\subsection*{Derivation of the Inertia Matrix $\mathbf{M}(q)$}

To compute the joint torques due to link and payload inertia, we derive the inertia matrix $\mathbf{M}(q)$ using the Lagrangian formulation. This derivation includes the effects of both links and a payload at the end-effector.

\subsubsection*{System Assumptions}

\begin{itemize}
    \item Link 1: mass $m_1$, length $l_1$, CoM at $l_1/2$, inertia $I_1 = \frac{1}{3} m_1 l_1^2$
    \item Link 2: mass $m_2$, length $l_2$, CoM at $l_2/2$, inertia $I_2 = \frac{1}{3} m_2 l_2^2$
    \item Payload: point mass $m_p$ at the end-effector
\end{itemize}

\subsubsection*{Velocity of Centers of Mass}

The squared velocities of the centers of mass and payload contribute to the kinetic energy:

\begin{align*}
\|\dot{\vec{r}}_{c1}\|^2 &= \left( \frac{l_1}{2} \dot{q}_1 \right)^2 \\
\|\dot{\vec{r}}_{c2}\|^2 &= l_1^2 \dot{q}_1^2 + \frac{l_2^2}{4} (\dot{q}_1 + \dot{q}_2)^2 + l_1 l_2 \cos(q_2) \dot{q}_1 (\dot{q}_1 + \dot{q}_2) \\
\|\dot{\vec{r}}_p\|^2 &= l_1^2 \dot{q}_1^2 + l_2^2 (\dot{q}_1 + \dot{q}_2)^2 + 2 l_1 l_2 \cos(q_2) \dot{q}_1 (\dot{q}_1 + \dot{q}_2)
\end{align*}

\subsubsection*{Kinetic Energy Expression}

The total kinetic energy $T$ is:

\begin{align*}
T &= \frac{1}{2} I_1 \dot{q}_1^2 
+ \frac{1}{2} m_1 \left( \frac{l_1}{2} \dot{q}_1 \right)^2 \\
&\quad + \frac{1}{2} I_2 (\dot{q}_1 + \dot{q}_2)^2
+ \frac{1}{2} m_2 \left[ l_1^2 \dot{q}_1^2 + \frac{l_2^2}{4} (\dot{q}_1 + \dot{q}_2)^2 + l_1 l_2 \cos(q_2) \dot{q}_1 (\dot{q}_1 + \dot{q}_2) \right] \\
&\quad + \frac{1}{2} m_p \left[ l_1^2 \dot{q}_1^2 + l_2^2 (\dot{q}_1 + \dot{q}_2)^2 + 2 l_1 l_2 \cos(q_2) \dot{q}_1 (\dot{q}_1 + \dot{q}_2) \right]
\end{align*}

\subsubsection*{From Kinetic Energy to Inertia Matrix}

The kinetic energy of a robotic system can always be written in a quadratic form with respect to joint velocities:

\[
T = \frac{1}{2}
\begin{bmatrix}
\dot{q}_1 & \dot{q}_2
\end{bmatrix}
\mathbf{M}(q)
\begin{bmatrix}
\dot{q}_1 \\
\dot{q}_2
\end{bmatrix}
\]

\noindent where:
\begin{itemize}
    \item $\dot{q}_i$ are the joint velocities,
    \item $\mathbf{M}(q)$ is the inertia matrix — a symmetric, positive definite matrix that depends on the geometry and mass distribution of the system.
\end{itemize}

\paragraph{How to Extract $\mathbf{M}(q)$:}
To find $\mathbf{M}(q)$, we compute the total kinetic energy $T$ and group the terms based on joint velocity products:
\begin{itemize}
    \item Coefficients of $\dot{q}_1^2$ go into $M_{11}$,
    \item Coefficients of $\dot{q}_1 \dot{q}_2$ and $\dot{q}_2 \dot{q}_1$ go into $M_{12} = M_{21}$,
    \item Coefficients of $\dot{q}_2^2$ go into $M_{22}$.
\end{itemize}

\paragraph{Example:}
Suppose the total kinetic energy is:

\[
T = A \dot{q}_1^2 + B \dot{q}_1 \dot{q}_2 + C \dot{q}_2^2
\]

\noindent This can be written in matrix form as:

\[
T = \frac{1}{2}
\begin{bmatrix}
\dot{q}_1 & \dot{q}_2
\end{bmatrix}
\begin{bmatrix}
2A & B \\
B & 2C
\end{bmatrix}
\begin{bmatrix}
\dot{q}_1 \\
\dot{q}_2
\end{bmatrix}
\]

\noindent Therefore:
\[
M_{11} = 2A, \quad M_{12} = M_{21} = B, \quad M_{22} = 2C
\]

\paragraph{Application to the 2-Link Manipulator:}
When we derived the full kinetic energy $T$ for the 2-link manipulator (links + payload), we had terms like:

\begin{itemize}
    \item $\frac{1}{2} I_1 \dot{q}_1^2$
    \item $\frac{1}{2} m_2 \left( l_1^2 \dot{q}_1^2 + l_1 l_2 \cos(q_2) \dot{q}_1 (\dot{q}_1 + \dot{q}_2) + \cdots \right)$
    \item $\frac{1}{2} m_p \left( \cdots + 2 l_1 l_2 \cos(q_2) \dot{q}_1 (\dot{q}_1 + \dot{q}_2) \right)$
\end{itemize}

We then expand these expressions and reorganize them by velocity terms:

\begin{itemize}
    \item Coefficients of $\dot{q}_1^2$ go into $M_{11}$
    \item Coefficients of $\dot{q}_1 \dot{q}_2$ into $M_{12}$
    \item Coefficients of $\dot{q}_2^2$ into $M_{22}$
\end{itemize}

\paragraph{conclusion:}
This standard Lagrangian technique allows us to extract the complete inertia matrix $\mathbf{M}(q)$ directly from the kinetic energy expression — capturing the dynamic coupling and inertial behavior of the robotic manipulator.
The inertia matrix M(q) is the matrix of coefficients of the \textbf{velocity squared and cross terms} in the total kinetic energy. 

Once obtained, $\mathbf{M}(q)$ is used in the full dynamic model:

\[
\mathbf{M}(q)\ddot{q} + \mathbf{C}(q, \dot{q})\dot{q} + \mathbf{G}(q) = \boldsymbol{\tau}
\]


\subsubsection*{Inertia Matrix $\mathbf{M}(q)$:}

\[
\mathbf{M}(q) =
\begin{bmatrix}
a + 2b \cos(q_2) & d + b \cos(q_2) \\
d + b \cos(q_2) & d
\end{bmatrix}
\]

where:
\begin{align*}
a &= I_1 + I_2 + m_1 \left(\frac{l_1^2}{4}\right) + m_2 \left(l_1^2 + \frac{l_2^2}{4} + l_1 l_2 \cos(q_2) \right) + m_p \left(l_1^2 + l_2^2 + 2 l_1 l_2 \cos(q_2)\right) \\
b &= m_2 \left(\frac{l_1 l_2}{2}\right) + m_p l_1 l_2 \\
d &= I_2 + m_2 \left(\frac{l_2^2}{4}\right) + m_p l_2^2
\end{align*}

\subsubsection*{Coriolis and Centrifugal Matrix $\mathbf{C}(q, \dot{q})$:}

\subsection*{Coriolis and Centrifugal Matrix $\mathbf{C}(q, \dot{q})$}

The Coriolis and centrifugal forces arise due to the velocity-dependent inertial effects in the manipulator. They account for how the motion of one joint affects the torque required at another joint due to dynamic coupling.

\paragraph{Dynamic Model Overview:}

The full dynamic model of the 2-link planar manipulator is written as:

\[
\mathbf{M}(q)\ddot{q} + \mathbf{C}(q, \dot{q})\dot{q} + \mathbf{G}(q) = \boldsymbol{\tau}
\]

\paragraph{Coriolis and Centrifugal Terms:}

The matrix $\mathbf{C}(q, \dot{q})$ is defined such that:

\[
\mathbf{C}(q, \dot{q})\dot{q} =
\begin{bmatrix}
C_{11} & C_{12} \\
C_{21} & C_{22}
\end{bmatrix}
\begin{bmatrix}
\dot{q}_1 \\
\dot{q}_2
\end{bmatrix}
\]

These terms come from the Christoffel symbols of the inertia matrix $\mathbf{M}(q)$. Each term \( C_{ij} \) is built using:

\[
C_{ijk} = \frac{1}{2} \left( \frac{\partial M_{ij}}{\partial q_k} + \frac{\partial M_{ik}}{\partial q_j} - \frac{\partial M_{jk}}{\partial q_i} \right)
\]

Then:

\[
C_{ij} = \sum_{k=1}^{2} C_{ijk} \dot{q}_k
\]

\paragraph{Simplified Form for 2-Link Manipulator:}

Using symbolic simplification for a 2-link planar robot with payload, the Coriolis and centrifugal terms reduce to:

\[
\mathbf{C}(q, \dot{q}) =
\begin{bmatrix}
-h \sin(q_2) \dot{q}_2 & -h \sin(q_2)(\dot{q}_1 + \dot{q}_2) \\
h \sin(q_2) \dot{q}_1 & 0
\end{bmatrix}
\]

\noindent where:

\[
h = m_2 \left( \frac{l_1 l_2}{2} \right) + m_p l_1 l_2
\]

\paragraph{Interpretation:}
\begin{itemize}
    \item The terms with $\dot{q}_1 \dot{q}_2$ are **Coriolis** terms — they result from the coupling of motions between joints.
    \item The terms with $\dot{q}_2^2$ are **centrifugal** terms — they act to "pull outward" when a joint rotates.
    \item These forces do not exist in static motion and become significant during fast, coupled joint movements.
\end{itemize}
  
\paragraph{Usage in Simulation and Control:}

These velocity-dependent torques must be compensated in:
\begin{itemize}
    \item High-speed motion planning
    \item Feedforward control design
    \item Accurate simulation of dynamic response
\end{itemize}

\[
\mathbf{C}(q, \dot{q}) =
\begin{bmatrix}
-h \sin(q_2) \dot{q}_2 & -h \sin(q_2)(\dot{q}_1 + \dot{q}_2) \\
h \sin(q_2) \dot{q}_1 & 0
\end{bmatrix}, \quad h = b
\]
\subsection*{Gravity Vector $\mathbf{G}(q)$}

The gravity vector $\mathbf{G}(q)$ represents the torques at the joints required to counteract the weight of the links and payload due to gravity. These torques act along the axis of each joint and must be compensated by the motors during any static or dynamic task.

\subsubsection*{Potential Energy of the System}

Let $g$ denote gravitational acceleration (typically $9.81~\text{m/s}^2$). We compute the total potential energy $V$ from the vertical positions of the centers of mass (CoM) of each link and the payload.

\paragraph{Vertical positions:}

\begin{align*}
y_{c1} &= \frac{l_1}{2} \sin(q_1) \\
y_{c2} &= l_1 \sin(q_1) + \frac{l_2}{2} \sin(q_1 + q_2) \\
y_p &= l_1 \sin(q_1) + l_2 \sin(q_1 + q_2)
\end{align*}

\paragraph{Total potential energy:}

\[
V = m_1 g y_{c1} + m_2 g y_{c2} + m_p g y_p
\]

\subsubsection*{Gravity Torques from Potential Energy}

The gravity vector is obtained by taking partial derivatives of $V$ with respect to the joint angles:

\[
G_i = \frac{\partial V}{\partial q_i} \quad \text{for } i = 1, 2
\]

Thus, the gravity torque vector is:

\[
\mathbf{G}(q) =
\begin{bmatrix}
\frac{\partial V}{\partial q_1} \\
\frac{\partial V}{\partial q_2}
\end{bmatrix}
\]

\subsubsection*{Final Expressions for Gravity Torques}

After differentiating:

\begin{align*}
G_1 &= g \left[
\frac{m_1 l_1}{2} \cos(q_1) 
+ m_2 \left( l_1 \cos(q_1) + \frac{l_2}{2} \cos(q_1 + q_2) \right)
+ m_p \left( l_1 \cos(q_1) + l_2 \cos(q_1 + q_2) \right)
\right] \\
G_2 &= g \left[
\frac{m_2 l_2}{2} \cos(q_1 + q_2)
+ m_p l_2 \cos(q_1 + q_2)
\right]
\end{align*}

\noindent So the gravity vector becomes:

\[
\mathbf{G}(q) =
\begin{bmatrix}
G_1 \\
G_2
\end{bmatrix}
\]

\subsubsection*{Interpretation:}
\begin{itemize}
    \item $G_1$ includes all gravitational torques about joint 1, from all masses.
    \item $G_2$ includes only the components acting through joint 2.
    \item These torques are essential for holding static positions and must be counteracted by motor torque even when the robot is not moving.
\end{itemize}


\subsubsection*{Gravity Vector $\mathbf{G}(q)$:}

\[
\mathbf{G}(q) =
\begin{bmatrix}
g \left( \frac{m_1 l_1}{2} + m_2 l_1 + m_2 \frac{l_2}{2} \cos(q_2) + m_p l_1 + m_p l_2 \cos(q_2) \right) \sin(q_1 + q_2) \\
g \left( \frac{m_2 l_2}{2} + m_p l_2 \right) \sin(q_1 + q_2)
\end{bmatrix}
\]

\section*{Joint Torque Calculation Over Motion}

To compute the required torques at each joint during motion from an initial position $q_{\text{start}}$ to a final position $q_{\text{end}}$ within time $t$, we use the full dynamic model:

\[
\boldsymbol{\tau}(t) = \mathbf{M}(q) \ddot{q}(t) + \mathbf{C}(q, \dot{q}) \dot{q}(t) + \mathbf{G}(q)
\]

\subsection*{Trajectory Assumptions}

We assume:
\begin{itemize}
    \item Motion starts and ends at rest: $\dot{q}(0) = \dot{q}(t) = 0$
    \item The trajectory follows a smooth acceleration profile, such as a trapezoidal or minimum-jerk trajectory
    \item For approximation, we can model acceleration as constant: $\ddot{q} \approx \frac{4 \Delta q}{t^2}$ for rest-to-rest motion over time $t$
\end{itemize}

\subsection*{Angular Displacement}

\[
\Delta q_i = q_{i,\text{final}} - q_{i,\text{start}} \quad \text{for } i = 1,2
\]

\subsection*{Angular Acceleration Estimate}

For a symmetric trapezoidal trajectory:

\[
\ddot{q}_i = \frac{4 \Delta q_i}{t^2}, \quad \dot{q}_i = \frac{2 \Delta q_i}{t} \quad \text{(peak velocity)}
\]

\subsection*{Torque Expression at Peak Load}

\[
\tau_i = \underbrace{M_{i1} \ddot{q}_1 + M_{i2} \ddot{q}_2}_{\text{Inertial torque}}
+ \underbrace{C_{i1} \dot{q}_1 + C_{i2} \dot{q}_2}_{\text{Coriolis and centrifugal torque}}
+ \underbrace{G_i}_{\text{Gravity torque}}
\]

\noindent All terms are evaluated at the **intermediate configuration** (e.g., midpoint) of the trajectory, which is where torques typically peak.


\subsection*{Total Torque Expressions}

Now, compute the total torques at each joint:

\begin{align*}
\tau_1 &= M_{11} \ddot{q}_1 + M_{12} \ddot{q}_2 
+ (-h \sin q_2) \dot{q}_2 \cdot \dot{q}_1 + (-h \sin q_2)(\dot{q}_1 + \dot{q}_2) \cdot \dot{q}_2 
+ G_1 \\
\tau_2 &= M_{12} \ddot{q}_1 + M_{22} \ddot{q}_2 
+ h \sin q_2 \cdot \dot{q}_1 \cdot \dot{q}_1 
+ G_2
\end{align*}
This is the Final expression for torque to be generated by motors in respective positions.

\vspace{1em}
\noindent\rule{\linewidth}{0.6pt}
\vspace{1em}


\section{Motor Selection Based on Torque Requirements} \label{sec:motors}

The motors at each joint must be capable of delivering the required torque throughout the manipulator's motion. This includes inertial torque, Coriolis/centrifugal effects, and gravitational compensation — especially under the worst-case payload condition of $m_p = 200$ g.

\subsection*{Torque Requirement Summary}

From Section~\ref{sec:torque}, the total required torque at each joint is given by:

\[
\tau_i = \underbrace{\sum_{j=1}^{2} M_{ij}(q) \ddot{q}_j}_{\text{Inertial Torque}} 
+ \underbrace{\sum_{j=1}^{2} C_{ij}(q, \dot{q}) \dot{q}_j}_{\text{Coriolis/Centrifugal Torque}} 
+ \underbrace{G_i(q)}_{\text{Gravitational Torque}}
\]

We take the torque values computed using the worst-case configuration (maximum payload and full horizontal reach) as the basis for motor selection.

\subsection*{Selection Criteria}

The selected motors must satisfy the following requirements:

\begin{itemize}
    \item \textbf{Torque Capability:} Continuous torque rating $\tau_{\text{rated}}$ must be greater than the maximum required torque:
    \[
    \tau_{\text{rated}} \geq \tau_{\text{max}} \cdot \text{SF}
    \]
    where SF is a safety factor (typically 1.5 to 2.0).
    
    \item \textbf{Speed Capability:} The motor must also support the required angular speed:
    \[
    \omega_i = \max(\dot{q}_i)
    \]

    \item \textbf{Form Factor and Weight:} Motor size and weight must be acceptable, especially for joint 2 where added mass affects dynamic load on joint 1.

    \item \textbf{Control Precision:} High-resolution encoders or integrated feedback may be required for accurate joint positioning and smooth motion.
\end{itemize}

\subsection*{Recommended Torque Margin Calculation}

Let:

\begin{align*}
\tau_{1,\text{max}} &= \text{Maximum evaluated torque at joint 1 with } m_p = 0.2~\text{kg} \\
\tau_{2,\text{max}} &= \text{Maximum evaluated torque at joint 2 with } m_p = 0.2~\text{kg}
\end{align*}

Then select motors such that:

\[
\tau_{1,\text{rated}} \geq \tau_{1,\text{max}} \cdot \text{SF}, \quad
\tau_{2,\text{rated}} \geq \tau_{2,\text{max}} \cdot \text{SF}
\]

\subsection*{Drive System Note}

If gearboxes are used, the required output torque is reduced by the gear ratio $G$, and motor torque is:

\[
\tau_{\text{motor}} = \frac{\tau_{\text{joint}}}{G}
\]

But this also reduces speed and may introduce backlash.

\subsection*{Motor Type Considerations}

\begin{itemize}
    \item \textbf{Joint 1:} Requires higher torque; typically a high-torque servo motor or NEMA stepper with gear reduction.
    \item \textbf{Joint 2:} Lower torque; can use compact servo or geared micro motor.

\end{itemize}

\vspace{1em}
\noindent\rule{\linewidth}{0.6pt}
\vspace{1em}

\section{Effect of Unexpected Overload} \label{sec:overload}

In practical operation, the manipulator may unintentionally attempt to lift objects that exceed the expected payload range (50–200 g). This section analyzes how such an overload affects the joint dynamics, actuator limits, and system safety.

\subsection*{Overload Scenario}

Let the expected payload be $m_p \in [0.05, 0.2]$ kg. Consider a case where the arm encounters a significantly heavier object, for example:
\[
m_p^{\text{actual}} = 0.4~\text{kg} \quad (\text{2$\times$ overload})
\]

This mass was not included in the motor selection or torque profiling. As a result, the actual torque demands at both joints increase beyond nominal.

\subsection*{Torque Amplification}

From the dynamic torque equation:

\[
\boldsymbol{\tau} = \mathbf{M}(q)\ddot{q} + \mathbf{C}(q, \dot{q})\dot{q} + \mathbf{G}(q)
\]

The gravitational torque increases linearly with payload:

\[
\Delta \tau_{G} \propto m_p \cdot l_2 \cdot \cos(q_1 + q_2)
\]

Similarly, inertial terms scale with $m_p$ during acceleration.

\subsection*{Implications}

\begin{itemize}
    \item \textbf{Motor saturation:} If $\tau_{\text{actual}} > \tau_{\text{rated}}$, the motor cannot track the desired motion → trajectory error or stalling.
    \item \textbf{Thermal stress:} Prolonged high torque increases motor temperature, risking damage.
    \item \textbf{Mechanical strain:} Joints, mounts, and gearbox could exceed design loads.
    \item \textbf{Controller instability:} If not properly bounded, large position or velocity errors could destabilize a PID or feedforward controller.
\end{itemize}

\subsection*{Safety and Mitigation Strategies}

\begin{itemize}
    \item \textbf{Torque sensors or current monitoring} to detect excessive loads in real time.
    \item \textbf{Emergency stop thresholds} for motor current or joint torque.
    \item \textbf{Compliance control (impedance/admittance)} to absorb shocks and prevent rigid responses to overloads.
    \item \textbf{Overload detection logic} that compares estimated vs. commanded torque and switches to a fail-safe state.
\end{itemize}

\vspace{1em}
\noindent\rule{\linewidth}{0.6pt}
\vspace{1em}

\section{Energy Efficiency Optimization} \label{sec:efficiency}

Minimizing energy consumption in robotic manipulators is critical for performance, battery life, thermal stability, and actuator longevity. This section develops a formal basis for optimizing energy usage and presents the most effective strategy for doing so in a 2-link arm: \textbf{minimum-torque trajectory optimization}.

\subsection*{1. Total Mechanical Energy Cost}

The total actuator energy is estimated by the integral of mechanical power over time:

\[
E = \int_0^T \boldsymbol{\tau}^T(t) \cdot \dot{\boldsymbol{q}}(t)~dt
\]

Substituting full dynamics:

\[
\boldsymbol{\tau}(t) = \mathbf{M}(q)\ddot{q} + \mathbf{C}(q, \dot{q})\dot{q} + \mathbf{G}(q)
\]

Then,

\[
E = \int_0^T \left[ \dot{q}^T \mathbf{M}(q)\ddot{q} + \dot{q}^T \mathbf{C}(q, \dot{q})\dot{q} + \dot{q}^T \mathbf{G}(q) \right] dt
\]

\noindent These are interpreted as:
\begin{itemize}
    \item \textbf{Inertial Power}: required to accelerate links and payload
    \item \textbf{Coriolis Power}: power loss due to joint coupling and inertial interactions
    \item \textbf{Gravitational Power}: work done against gravity
\end{itemize}

\subsection*{2. Optimization Objective}

The energy-optimal trajectory problem is posed as:

\[
\min_{q(t)} \int_0^T \| \boldsymbol{\tau}(t) \|^2 dt
\quad \text{subject to:}
\begin{cases}
q(0) = q_{\text{start}}, \quad q(T) = q_{\text{end}} \\
\dot{q}(0) = \dot{q}(T) = 0
\end{cases}
\]

This results in a nonlinear trajectory optimization problem over $q(t)$, often solved using:
\begin{itemize}
    \item Euler-Lagrange or Pontryagin’s minimum principle
    \item Direct collocation (e.g., using optimization libraries)
    \item Time-scaling techniques using polynomial interpolation
\end{itemize}

\subsection*{3. Best Method: Minimum Torque-Squared Trajectory}

Define the cost function:

\[
J = \int_0^T \boldsymbol{\tau}^T(t) \boldsymbol{\tau}(t) ~dt
\]

This is minimized when:
\begin{itemize}
    \item Trajectory is smooth (low $\ddot{q}$ and $\dot{q}$)
    \item Joint velocities and accelerations are scaled according to dynamic coupling
    \item Payload and mass distribution are considered in $\mathbf{M}(q)$ and $\mathbf{G}(q)$
\end{itemize}

\subsubsection*{Implementation Outline}

\begin{enumerate}
    \item Choose a time-scaling function $s(t) \in [0, 1]$ with fixed duration $T$
    \item Represent $q(t) = q_{\text{start}} + \Delta q \cdot f(s(t))$ using $f(s)$ (e.g., 5th-order polynomial)
    \item Plug $q(t)$, $\dot{q}(t)$, $\ddot{q}(t)$ into torque expression
    \item Minimize $J$ numerically over coefficients of $f(s)$ using optimization tools
\end{enumerate}

\subsection*{4. Real-Time Adaptation (Optional)}

If onboard sensing estimates payload $m_p$, the planner can:
\begin{itemize}
    \item Adapt $T$ (duration) to reduce peak torque
    \item Re-optimize trajectory to minimize energy for detected $m_p$
    \item Switch to energy-conserving low-torque control mode
\end{itemize}

The most effective method to reduce energy usage in a 2-link manipulator is to use \textbf{minimum-torque-squared trajectory optimization}, adjusted in real time based on sensed system parameters. This accounts for full system dynamics and adapts to varying mass and geometry for maximum efficiency.

\vspace{1em}
\noindent\rule{\linewidth}{0.6pt}
\vspace{1em}


\end{document}
